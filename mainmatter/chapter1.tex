\chapter{Introduction}
\label{chap:intro}

\section{Cancer is a disease of the genome and epigenome}

Cancer is one of the largest causes of death worldwide, ranking in the top ten most frequent causes in over 150 countries and most frequent in over 40 \cite{brayGlobalCancerStatistics2018}.
Disease treatment is complicated by the fact that cancers are a myriad of diseases with unique origins, symptoms, and treatment options, often related to the cell of origin.
However, numerous hallmarks of cancers have emerged over the last 50 years to provide understanding about what biological aberrations cause tumours to initiate, how they develop over time, and how they respond to therapeutic interventions \cite{hanahanHallmarksCancer2000,hanahanHallmarksCancerNext2011,flavahanEpigeneticPlasticityHallmarks2017,pavlovaEmergingHallmarksCancer2016} (\Cref{fig:intro_hallmarks}).

\newflexfigure{chapter1/hallmarks.png}{The hallmarks of cancer}{Adapted from \cite[REF][]{hanahanHallmarksCancerNext2011}.}{fig:intro_hallmarks}{0.5\textwidth}

Many of these hallmarks of cancer can be achieved through aberrations to the molecular machinery that enables cells to function normally.
For example, genome instability can be achieved by inhibiting DNA repair machinery, as is observed with abnormalities in \emph{MLH1} and \emph{MSH2} repair genes in colorectal cancers \cite{lengauerGeneticInstabilitiesHuman1998} or mutations to \emph{BRCA1}, \emph{BRCA2}, and \emph{ATM} genes in \gls{pca} \cite{abeshouseMolecularTaxonomyPrimary2015}.
Similarly, replicative immortality can be achieved through telomere elongation by over-expression of the \emph{TERT} gene \cite{vinagreFrequencyTERTPromoter2013}.
Mutations to the \emph{TERT} promoter, resulting in its over-expression, were first identified in melanomas \cite{huangHighlyRecurrentTERT2013,hornTERTPromoterMutations2013}, but have since been further identified in bladder, thyroid, and brain cancers \cite{vinagreFrequencyTERTPromoter2013,nagarajanRecurrentEpimutationsActivate2014,sternMutationTERTPromoter2015}.
But while cancer has long been viewed as a disease of the genome \cite{hanahanHallmarksCancer2000}, there are many avenues cells can take to arrive at any of these hallmarks resulting from aberrations of how genes are expressed inside the cell nucleus.

Genes, encoded as DNA, are transcribed into \gls{mrna}, which are then translated into proteins, in the process known as the central dogma of molecular biology \cite{albertsMolecularBiologyCell2015} (\Cref{fig:intro_dogma}a).
The transcription of genes into \gls{mrna} requires RNA polymerase to bind at \glspl{tss} located within gene promoters \cite{goodrichUnexpectedRolesCore2010}.
The recruitment of RNA polymerase is aided by a special class of proteins, termed \glspl{tf}, that can bind at DNA sequences either close to a gene's promoter, or far from it at DNA elements termed enhancers, insulators, and silencers \cite{schoenfelderLongrangeEnhancerPromoter2019,spitzTranscriptionFactorsEnhancer2012,ongEnhancerFunctionNew2011,anderssonDeterminantsEnhancerPromoter2020,gasznerInsulatorsExploitingTranscriptional2006,oudelaarRelationshipGenomeStructure2020}.
These different DNA elements can be tens to thousands of \glspl{bp} apart from each other, but the DNA polymer bends to take up a small space inside the nucleus, and this can bring distal DNA elements close together in three-dimensional space \cite{finnMolecularBasisBiological2019,zhouChartingHistoneModifications2011} (\Cref{fig:intro_dogma}b).
The ability of \glspl{tf} to bind at certain DNA elements is dependent on multiple features of the DNA inside the nucleus, including its sequence, its shape, and nearby chemical modifications, such as \gls{dname} or modifications to the histone proteins that comprise the nucleosomes that DNA is wrapped around \cite{zhuTranscriptionFactorsReaders2016,fureyChIPSeqNew2012,carterEpigeneticBasisCellular2021}.
While the DNA sequence remains consistent across all cells in an organism, the chromatin state can vary, allowing for cell-type-specific gene expression.
These chromatin states can be aberrantly displayed in cell types that do not normally have that state, which can lead to the aberrant over- or under-expression of genes in those cells.

\newfigure{chapter1/transcription-dogma.png}{The basics of gene expression inside the nucleus.}{\textbf{a.} The central dogma of molecular biology. \textbf{b.} Schematic of the transcription machinery to initiate transcription. Both panels are adapted from \cite[REF][]{albertsMolecularBiologyCell2015}.}{fig:intro_dogma}


\subsection{}

\section{Dissertation structure}

I begin with \Cref{chap:FOXA1} by exploring the \emph{cis}-regulatory landscape of \gls{pca} and delineating the \glspl{cre} of the prostate oncogene \emph{FOXA1}.
I demonstrate the essentiality of \emph{FOXA1} for prostate tumours, identify putative \glspl{cre} based on integration of multiomic datasets in \gls{pca} cell lines, and assess the functional impact of recurrent \gls{pca} \glspl{snv} on \emph{FOXA1} expression and \gls{tf} binding.

With the \emph{cis}-regulatory network of \emph{FOXA1} established in \gls{pca}, I attempt to construct the \emph{cis}-regulatory landscape genome-wide in \gls{pca} with \gls{3c} mapping in \Cref{chap:3D}.
Using Hi-C, I characterize the three-dimensional chromatin organization of \gls{pca} and investigate the relationship between chromatin organization, \glspl{sv}, and the hijacking of \emph{cis}-regulatory networks more generally.

In assessing the impact of \glspl{sv} on chromatin organization, I uncovered a statistical problem stemming from the lack of recurrent \glspl{sv} across \gls{pca} patients, leading to unbalanced experimental comparisons.
To address this problem, I developed a statistical method for reducing error in gene expression fold-change estimates from unbalanced experimental designs in \Cref{chap:JS} and characterize the method.

Given the shared importance of mutations to \glspl{tf} and epigenetic enzymes in prostate cancer and leukemias, in \Cref{chap:BALL} I explore the epigenetic landscape of \gls{ball} and its relapse after treatment.
I characterize molecular changes to \gls{ball} tumours over the course of disease relapse and identify important changes to \gls{dname} that indicate the reversion to a stem-like phenotype, often present in a subpopulation of cells at diagnosis.

Together, this dissertation investigates the multiple layers of the chromatin architecture that contribute to oncogenesis and cancer progression.
I demonstrate that aberrations to the genome, epigenome, and three-dimensional organization of chromatin play important roles individually, and together, in the orchestration of the disease.
