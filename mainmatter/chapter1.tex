\chapter{Introduction}
\label{chap:intro}

\section{Cancer is a disease of the genome and epigenome}

Cancer is one of the largest causes of death worldwide, ranking in the top ten most frequent causes in over 150 countries and most frequent in over 40 \cite{brayGlobalCancerStatistics2018}.
Disease treatment is complicated by the fact that cancers are a myriad of diseases with unique origins, symptoms, and treatment options.
However, numerous hallmarks of cancers have emerged over the last 50 years to provide understanding about what biological aberrations cause tumours to initiate, how they develop over time, and how they respond to therapeutic interventions \cite{hanahanHallmarksCancer2000,hanahanHallmarksCancerNext2011,flavahanEpigeneticPlasticityHallmarks2017,pavlovaEmergingHallmarksCancer2016}.

\subsection{}

\section{Dissertation structure}

I begin with \Cref{chap:FOXA1} by exploring the \emph{cis}-regulatory landscape of \gls{pca} and delineating the \glspl{cre} of the prostate oncogene \emph{FOXA1}.
I demonstrate the essentiality of \emph{FOXA1} for prostate tumours, identify putative \glspl{cre} based on integration of multiomic datasets in \gls{pca} cell lines, and assess the functional impact of recurrent \gls{pca} \glspl{snv} on \emph{FOXA1} expression and \gls{tf} binding.

With the \emph{cis}-regulatory network of \emph{FOXA1} established in \gls{pca}, I attempt to construct the \emph{cis}-regulatory landscape genome-wide in \gls{pca} with \gls{3c} mapping in \Cref{chap:3D}.
Using Hi-C, I characterize the three-dimensional chromatin organization of \gls{pca} and investigate the relationship between chromatin organization, \glspl{sv}, and the hijacking of \emph{cis}-regulatory networks more generally.

In assessing the impact of \glspl{sv} on chromatin organization, I uncovered a statistical problem stemming from the lack of recurrent \glspl{sv} across \gls{pca} patients, leading to unbalanced experimental comparisons.
To address this problem, I developed a statistical method for reducing error in gene expression fold-change estimates from unbalanced experimental designs in \Cref{chap:JS} and characterize the method.

Given the shared importance of mutations to \glspl{tf} and epigenetic enzymes in prostate cancer and leukemias, in \Cref{chap:BALL} I explore the epigenetic landscape of \gls{ball} and its relapse after treatment.
I characterize molecular changes to \gls{ball} tumours over the course of disease relapse and identify important changes to \gls{dname} that indicate the reversion to a stem-like phenotype, often present in a subpopulation of cells at diagnosis.

Together, this dissertation investigates the multiple layers of the chromatin architecture that contribute to oncogenesis and cancer progression.
I demonstrate that aberrations to the genome, epigenome, and three-dimensional organization of chromatin play important roles individually, and together, in the orchestration of the disease.
