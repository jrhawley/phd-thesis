\chapter{Discussion \& Future Directions}
\label{chap:discussion}

Each of the previous chapters have presented a story interrogating multiple components of the chromatin architecture, how they interact with each other, and the plethora of computational and experimental methods required to unravel this architecture.
\Cref{chap:FOXA1} identifies and validates \glspl{cre} of the \emph{FOXA1} gene, a critical \gls{tf} that regulates \gls{pca} development and regulates \gls{ar} expression to control disease progression.
\Cref{chap:3D} expands on these ideas to investigate how the three-dimensional genome organization impacts gene regulatory networks and how genetic aberrations can alter this organization to promote oncogenesis.
\Cref{chap:JS} develops a mathematical and computational framework to reduce uncertainty about how individual aberrations in chromatin architecture impact gene expression.
Finally, \Cref{chap:BALL} identifies the strong relationship between genetic and epigenetic profiles in \gls{ball} relapse and investigates how \gls{dname} changes and revision to a more stem-like chromatin state may underlie disease relapse.
Together, the work presented in this thesis demonstrates that different components of the chromatin architecture, the genome, molecular chromatin modifications, and three-dimensional organization, can all individually contribute to cancer development and progression.
Moreover, this thesis demonstrates that aberrations in these components work together to drive disease.
These multiple components of the chromatin architecture need to be studied in tandem to understand the origins of cancer and how to develop curative treatments for it.

\section{Implications of non-coding \glsentrylongpl{snv} targeting a single gene}

In \Cref{chap:FOXA1},

\begin{itemize}
  \item using essentiality from multiple cell lines to converge on uniformly important \glspl{tf} in a given tumour type
  \item idea that \glspl{snv} converge on \glspl{cre} of important genes in a given tumour type
  \item other aberrations, such as \gls{dname}, histone modifications, complex \glspl{sv}, or chromatin accessibility may show similar impacts on the expression of these genes in other cancer types
  \item experimental and computational methods to identify and validate \glspl{cre} for a given gene
  \item importance of considering the set of \glspl{cre} for a given gene, or plexus, when considering the impact of chromatin aberrations in oncogenesis or relapse
  \item impact on "driver" identification in cancers
  \item all of the above extend beyond \gls{pca} and can apply in all cancers or diseases with known chromatin aberrations
\end{itemize}

\section{Implications of three-dimensional organization and enhancer hijacking in \glsentrylong{pca}}

\begin{itemize}
  \item optimized low-input Hi-C method to interrogate genome organization in more tissues than previously possible
  \item largest collection of 3D organization data in prostate tumours to date
  \item add to evidence that \glspl{sv} can, but rarely, alter 3D structure in disease, prompting further interrogation into how and why this appears to be the case
  \item motivates use of microscopy measurements in single cells to see how structure is altered and if sub-clonal mutations interfere with the ability to detect changes in structure
  \item statistical methods may make progress to help identify effect of individual, non-recurrent events, such as in \Cref{chap:JS}, as well as computational libraries for developing better tooling in this area
  \item alternatively further push for development in organoid models that can replicate the chromatin state of tumour tissue from small input samples
  \item use genome organization methods to fully characterize GRNs for each individual gene (both first- and higher-order, depending on how much perturbations to these networks affect gene expression)
  \item consider evolutionary lens of GRN across species instead of just thinking of genes as single units
\end{itemize}

\section{Implications of \glsentrylong{dname} changes in relapse}

\begin{itemize}
  \item prioritize role of stem cells in disease relapse
  \item use increasing \gls{dname} as a potential biomarker of relapse
  \item may be able to use blood-based \gls{dname} detection to create a non-invasive test for this development
  \item may be able to treat \gls{ball} patients with de-methylating agents if gains in \gls{dname} are observed to prevent relapse
\end{itemize}

\section{Summary and concluding remarks}

\begin{itemize}
  \item work does not focus on a single disease, should extend this type of analysis to all cancers, since they all appear to harbour aberrations affecting multiple components of the chromatin architecture
  \item multi-pronged approach of computational, statistical, and molecular, and microscopy methods optimized for low-input samples targeting the set of DNA elements and their relationships to each other in individual patients to develop personalized medicines and treat cancer at its origins in the chromatin
\end{itemize}
