\section{Discussion}

In this work, we developed a statistical framework for differential expression analysis where unreplicated conditions are found.
We found that its larger bias and smaller variance, compared to conventional \gls{ols} estimators, can improve statistical inferences, particularly when focusing on $\le 100$ transcripts.
In cases with unreplicated conditions, DESeq2 \cite{loveModeratedEstimationFold2014}, edgeR \cite{robinsonEdgeRBioconductorPackage2010}, and Sleuth \cite{pimentelDifferentialAnalysisRNAseq2017,yiGenelevelDifferentialAnalysis2018} frameworks group all samples, both case and control, before performing normalization and performing variance shrinkage.
The method presented here builds on these shrinkage methods to reduce uncertainty in gene expression fold changes for scenarios where biological replicates are missing.
This method can help improve differential gene expression analyses for singular observations, such as unique chromatin aberrations or rare diseases.

Many statistical methods exist to similarly reduce uncertainty.
However, as has been shown here, balanced experimental designs, where the number of case and control samples are equal, outperformed both the \gls{ols} and \gls{js} methods.
This is similarly the case with other differential analysis methods \cite{schurchHowManyBiological2016,gierlinskiStatisticalModelsRNAseq2015}.
Complex statistical methods often cannot overcome limitations in sample size.
Thus, technological development for biological models that recapitulate the chromatin state of tissues, such as organoids and cell lines, may be more advantageous for experimental validation \cite{zanoniModelingNeoplasticDisease2020}.
The statistical method presented here may be useful in single-cell \gls{rnaseq} differential analysis, where single cells are often clustered together to improve differential expression analysis inferences \cite{chenAssessmentComputationalMethods2019}.
This method may similarly apply to single-cell methods of other epigenomic assays,  such as \gls{chipseq} \cite{rotemSinglecellChIPseqReveals2015}, \gls{cutrun} \cite{hainerProfilingPluripotencyFactors2019}, or \gls{atacseq} \cite{buenrostroSinglecellChromatinAccessibility2015}.
These applications require special consideration for inflated zero counts in single-cell data \cite{chenAssessmentComputationalMethods2019}, and require further investigation before adopting ideas presented here.
