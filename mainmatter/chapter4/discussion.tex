\section{Discussion}

\begin{enumerate}
  \item summary of the paper
  \begin{enumerate}
    \item introduced a new statistical method for dealing with $n$-of-1 scenarios for differential expression analysis
    \item showed that its higher bias and smaller variance, compared to the \gls{ols} method, can improve its differential results, especially when focusing on a few 100 transcripts or less
    \item this makes it a good method for investigating differential expression in a few scenarios: rare diseases (see https://ojrd.biomedcentral.com/articles/10.1186/s13023-018-0765-y, https://www.disgenet.org/downloads, https://www.orpha.net/consor/cgi-bin/index.php, and https://www.omim.org/), retrospective studies, and preliminary analyses
    \item this tool has been packaged into an R package
  \end{enumerate}
  \item similar approaches for $n$-of-1 scenarios
  \begin{enumerate}
    \item contrast with DESeq2 approach
    \item contrast with edgeR approach
    \item mention how the evenly split design achieved higher results overall, and that this is not a replacement for good experimental design
  \end{enumerate}
  \item possible future directions for work in this area
  \begin{enumerate}
    \item applications in single-cell RNA-seq differential analysis
    \item functional extensions to other epigenomic assays, such as CUT\&RUN of ATAC-seq, also with their single-cell extensions
    \item statistical extensions to GLMs in other genomics applications
  \end{enumerate}
\end{enumerate}
