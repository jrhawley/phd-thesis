\section{Introduction}

\begin{itemize}
  \item the two main approaches for reducing error in a model are to reduce the model variance or model bias \Cref{fig:JS_fig1}
  \item here we attempt to decrease \gls{mse} by simultaneously increasing the bias and decreasing the variance in fold change coefficient estimators
  \item derivation for the \gls{js} estimator can be found in \Cref{sec:JS_derivation}
  \item Equation for the \gls{js} estimator can be seen in \Cref{fig:JS_fig1}b. 
  \item in theory, this may increase the error of some transcripts, but will decrease \gls{mse} for a set of transcripts in aggregate \Cref{sec:JS_moments}
\end{itemize}

\newfigure{chapter4/Figure1.png}{Reducing the bias-variance tradeoff by combining information across multiple features}{\textbf{a.} Schematic of the bias-variance tradeoff for assessing model performance. Dartboard on the left shows low bias of the darts (mean is close to the bullseye) but a large variance. Dartboard on the right shows a high bias of the darts (mean is off-centre), but a small variance. \textbf{b.} For a $p$-variate normal distribution from which a single observation is made, the naive estimator has a higher \gls{mse} than the \gls{js} estimator, defined as $\hat{\mu}^{(2)}$. \textbf{c.} An analogy showing how the \gls{js} estimators work in theory. Trying to estimate the mean height, weight, and age for the entire population ($\mu$) from a single person will give an estimate that is likely far from the truth ($\hat{\mu}^{(1)}$). Combining information from the three variables together can produce an estimate that is closer to the truth ($\hat{\mu}^{(2)}$).}{fig:JS_fig1}

First, we derive the \gls{js} estimator fold gene expression fold change and relate it to the \gls{ols} estimator.
Then, using simulations from a highly replicated \gls{rnaseq} experiment \cite{gierlinskiStatisticalModelsRNAseq2015}, we compare the differences in statistical inferences between the \gls{js} and \gls{ols} estimators.
Finally, we investigate how the number of transcripts under consideration affects the reduction in \gls{mse}, suggesting how this method can be best used in practice.
