\section{Methods}

\subsection{\glsentrylong{rnaseq} data collection and pre-processing}

\Gls{rnaseq} reads were obtained from \cite[REF][]{gierlinskiStatisticalModelsRNAseq2015} from the \gls{sra} (Project Accession PRJEB5348).
Briefly, the $\Delta$\emph{Snf2} \gls{ko} and \gls{wt} cells were both sequenced as single-end, 50 bp reads, split across seven lanes of a single Illumina HiSeq 2000 flow cell.
The quality of the sequencing reads was assessed with FastQC \cite{andrewsFastQCQualityControl2010}.
The \emph{Saccharomyces cerevisiae} R64-1-1 reference transcriptome index from Ensembl (v96) was downloaded from https://github.com/pachterlab/kallisto-transcriptome-indices/.
\Gls{rnaseq} reads from all technical replicates of a single biological replicate were quantified against the reference transcriptome with Kallisto (v0.46.2) \cite{brayNearoptimalProbabilisticRNAseq2016} with the following command:

\begin{lstlisting}[basicstyle=\ttfamily]
kallisto quant --bootstrap-samples 100 --pseudobam --threads 8 --index /path/to/R64-1-1.idx --output-dir {output_dir} {input.fastq.gz} > {output_report}
\end{lstlisting}

\subsection{Differential expression analysis}

Differential expression analysis was conducted in R (v4.0.2) \cite{rcoreteamLanguageEnvironmentStatistical2013} with the Sleuth package (v0.30.0) \cite{pimentelDifferentialAnalysisRNAseq2017,yiGenelevelDifferentialAnalysis2018}.
Statistical significance for the fold change of transcripts from the mutation status was determined with a two-sided Wald test.
Multiple testing correction was performed using the Benjamini-Hochberg FDR method \cite{benjaminiControllingFalseDiscovery1995}.
Transcripts were determined to be significantly differentially expressed if $FDR < 0.01$.

This method was used for differential expression analysis for the full experimental design (48 $\Delta$\emph{Snf2} replicates and 48 \gls{wt} replicates), as well as all simulations with smaller sample sizes.

\subsection{Random sampling and simulations}

From the 48 biological replicates of the $\Delta$\emph{Snf2} and \gls{wt} samples, $N$ total samples were randomly selected, where $N \in \left\{4, 6, 8, ..., 22, 24 \right\}$.
The sampling was repeated 30 times, independently.
The samples were chosen according to of the following experimental designs:

\begin{enumerate}
  \item 30 iterations of $N/2$ $\Delta$\emph{Snf2} vs $N/2$ \gls{wt}, compared using the \gls{ols} estimator
  \item 15 iterations of 1 $\Delta$\emph{Snf2} vs $N - 1$ \gls{wt} + 15 iterations of $N - 1$ $\Delta$\emph{Snf2} vs 1 \gls{wt}, compared using the \gls{ols} estimator
  \item 15 iterations of 1 $\Delta$\emph{Snf2} vs $N - 1$ \gls{wt} + 15 iterations of $N - 1$ $\Delta$\emph{Snf2} vs 1 \gls{wt}, compared using the \gls{js} estimator
\end{enumerate}

Differential expression analysis was calculated with the Sleuth package \cite{pimentelDifferentialAnalysisRNAseq2017,yiGenelevelDifferentialAnalysis2018}.
For the \gls{js} calculations, the scaling coefficient, $c$, was set to its largest value of $2 \left( \frac{ \text{Tr} (\Sigma) }{\lambda_L} - 2 \right)$ to reduce error as much as possible.
For each simulation, the number of \gls{tp}, \gls{fp}, \gls{tn}, and \gls{fn} calls was calculated by comparing the classification of a given transcript as significantly differentially expressed or not (i.e. if $FDR < 0.01$ for a given transcript) in the simulation to the full experiment of 48 $\Delta$\emph{Snf2} vs 48 \gls{wt}.
With this confusion matrix, derived rates, such as the true positive rate, were then calculated.

\subsection{Random sampling of smaller numbers of transcripts}

In a similar fashion to above, 30 iterations of randomly selected samples were chosen to match the same three scenarios as above, with a fixed $N = 6$ (15 iterations with 1 $\Delta$\emph{Snf2} vs 5 \gls{wt} and 15 iterations with 5 $\Delta$\emph{Snf2} vs 1 \gls{wt}).
A subset of transcripts, $S$, were randomly selected from those transcripts with $\ge 10$ estimated reads in all 6 of the randomly selected samples.
Fold change estimates for the three scenarios were calculated as above.
The number of transcripts selected was chosen from $|S| \in \left\{3, 10, 25, 50, 100, 250 \right\}$.

\Gls{mse} was calculated by comparing the fold change estimate from each iteration to the fold change estimate from the full experiment with 48 $\Delta$\emph{Snf2} vs 48 \gls{wt}.
Percent differences in \gls{mse} were calculated as:

\begin{equation*}
  \text{Percent difference } = \frac{MSE_{JS} - MSE_{OLS}}{MSE_{OLS}}
\end{equation*}

\subsection{Code and data availability}

\Gls{rnaseq} data was obtained from the \gls{sra} (Project Accession PRJEB5348).
All code for pre-processing, analysis, and plotting can be found on GitHub (https://github.com/jrhawley/n1diff).
