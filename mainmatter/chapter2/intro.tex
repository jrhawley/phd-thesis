\section{Introduction}

Prostate cancer is the second most commonly diagnosed cancer among men with an estimated 1.3 million new cases worldwide in 2018 \cite{brayGlobalCancerStatistics2018}.
Although most men diagnosed with primary prostate cancer are treated with curative intent through surgery or radiation therapy, treatments fail in 30\% of patients within 10 years \cite{boorjianLongTermOutcomeRadical2007} resulting in a metastatic disease \cite{litwinDiagnosisTreatmentProstate2017}.
Patients with metastatic disease are typically treated with anti-androgen therapies, the staple of aggressive prostate cancer treatment \cite{attardProstateCancer2016}.
Despite the efficacy of these therapies, recurrence ultimately develops into lethal metastatic castration resistant prostate cancer (mCRPC) \cite{attardProstateCancer2016}.
As such, there remains a need to improve our biological understanding of prostate cancer development and find novel strategies to treat patients.
Sequencing efforts identified coding somatic single-nucleotide variants (SNVs) mapping to FOXA1 in up to 9\% \cite{abeshouseMolecularTaxonomyPrimary2015,fraserGenomicHallmarksLocalized2017,barbieriExomeSequencingIdentifies2012, grassoMutationalLandscapeLethal2012,paroliaDistinctStructuralClasses2019,adamsFOXA1MutationsAlter2019} and 13\% \cite{paroliaDistinctStructuralClasses2019,adamsFOXA1MutationsAlter2019,robinsonIntegrativeClinicalGenomics2015} of primary and metastatic castration resistant prostate cancer (mCRPC) patients, respectively.
These coding somatic SNVs target the Forkhead and transactivation domains of FOXA1 \cite{robinsonFOXA1MutationsHormonedependent2013}, altering its pioneering functions to promote prostate cancer development \cite{adamsFOXA1MutationsAlter2019,gaoForkheadDomainMutations2019}.
Outside of coding SNVs, whole genome sequencing also identified somatic SNVs and indels in the 3' UTR and C-terminus of FOXA1 in \\textapprox 12\% of mCPRC patients \cite{annalaFrequentMutationFOXA12018}.
In addition to SNVs, the FOXA1 locus is a target of structural rearrangements in both primary and metastatic prostate cancer tumours, inclusive of duplications, amplifications, and translocations \cite{paroliaDistinctStructuralClasses2019,adamsFOXA1MutationsAlter2019}.
Taken together, FOXA1 is recurrently mutated taking into account both its coding and flanking noncoding sequences across various stages of prostate cancer development.

FOXA1 serves as a pioneer transcription factor (TF) that can bind to heterochromatin, promoting its remodelling to increase accessibility for the recruitment of other TFs \cite{yangCurrentPerspectivesFOXA12015}.
FOXA1 binds to chromatin at cell-type specific genomic coordinates facilitated by the presence of mono- and dimethylated lysine 4 of histone H3 (H3K4me1 and H3K4me2) histone modifications \cite{lupienFoxA1TranslatesEpigenetic2008,eeckhouteCelltypeSelectiveChromatin2008}.
In prostate cancer, FOXA1 is known to pioneer and reprogram the binding of the Androgen Receptor (AR) alongside HOXB13 \cite{pomerantzAndrogenReceptorCistrome2015}.
Independent from its role in AR signalling, FOXA1 also regulates the expression of genes involved in cell cycle regulation in prostate cancer \cite{imamuraFOXA1PromotesTumor2012,imamuraFOXA1PromotesTumor2012,xuAndrogensInduceProstate2006}.
For instance, FOXA1 co-localizes with CREB1 to regulate the transcription of genes involved in cell cycle processes, nuclear division and mitosis in mCRPC \cite{imamuraFOXA1PromotesTumor2012,jinAndrogenReceptorIndependentFunction2013,xuAndrogensInduceProstate2006,yangFOXA1PotentiatesLineagespecific2016,zhangFOXA1DefinesCancer2016,augelloFOXA1MasterSteroid2011,sunkelIntegrativeAnalysisIdentifies2017}.
FOXA1 has also been shown to promote feed-forward mechanisms to drive disease progression \cite{niAmplitudeModulationAndrogen2013,sasseFeedforwardTranscriptionalProgramming2015}.
Hence, FOXA1 contributes to AR-dependent and AR-independent processes favouring prostate cancer development.

Despite the oncogenic roles of FOXA1, therapeutic avenues to inhibit its activity in prostate cancer are lacking.
In the breast cancer setting for instance, the use of cyclin-dependent kinases inhibitors have been suggested based on their ability to block FOXA1 activity on chromatin \cite{wangHighThroughputChemical2018}.
As such, understanding the governance of FOXA1 mRNA expression offers an alternative strategy to find modulators of its activity.
Gene expression relies on the interplay between distal cis-regulatory elements (CREs), such as enhancers and anchors of chromatin interaction, and their target gene promoter(s) \cite{rowleyOrganizationalPrinciples3D2018}.
These elements can lie tens to hundreds of kilobases (kbp) away from each other on the linear genome but physically engage in close proximity with each other in the three-dimensional space \cite{vernimmenHierarchyTranscriptionalActivation2015}.
By measuring contact frequencies between loci through the use of chromatin conformation capture-based technologies, it enables the identification of regulatory plexuses corresponding to sets of CREs in contact with each other \cite{sallariConvergenceDispersedRegulatory2016,baileyNoncodingSomaticInherited2016}.
By leveraging these technologies, we can begin to understand the three-dimensional organization of the prostate cancer genome and delineate the FOXA1 regulatory plexus.

Here, we integrate epigenetics and genetics from prostate cancer patients and model systems to delineate CREs establishing the regulatory plexus of FOXA1.
We functionally validate a set of six mutated CREs that regulate FOXA1 mRNA expression.
We further show that SNVs mapping to these CREs are capable of altering their transactivation potential, likely through modulating the binding of key prostate cancer TFs.
