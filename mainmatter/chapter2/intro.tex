\section{Introduction}

\Gls{pca} is the second most commonly diagnosed cancer among men with an estimated 1.3 million new cases worldwide in 2018 \cite{brayGlobalCancerStatistics2018}.
Although most men diagnosed with primary \gls{pca} are treated with curative intent through surgery or radiation therapy, treatments fail in 30\% of patients within 10 years \cite{boorjianLongTermOutcomeRadical2007} resulting in a metastatic disease \cite{litwinDiagnosisTreatmentProstate2017}.
Patients with metastatic disease are typically treated with anti-androgen therapies, the staple of aggressive \gls{pca} treatment \cite{attardProstateCancer2016}.
Despite the efficacy of these therapies, recurrence ultimately develops into lethal \gls{mcrpc} \cite{attardProstateCancer2016}.
As such, there remains a need to improve our biological understanding of \gls{pca} development and find novel strategies to treat patients.
Sequencing efforts identified coding somatic \glspl{snv} mapping to FOXA1 in up to 9\% \cite{abeshouseMolecularTaxonomyPrimary2015,fraserGenomicHallmarksLocalized2017,barbieriExomeSequencingIdentifies2012, grassoMutationalLandscapeLethal2012,paroliaDistinctStructuralClasses2019,adamsFOXA1MutationsAlter2019} and 13\% \cite{paroliaDistinctStructuralClasses2019,adamsFOXA1MutationsAlter2019,robinsonIntegrativeClinicalGenomics2015} of primary and metastatic \gls{pca} patients, respectively.
These coding somatic \glspl{snv} target the Forkhead and transactivation domains of FOXA1 \cite{robinsonFOXA1MutationsHormonedependent2013}, altering its pioneering functions to promote \gls{pca} development \cite{adamsFOXA1MutationsAlter2019,gaoForkheadDomainMutations2019}.
Outside of coding \glspl{snv}, \gls{wgs} also identified somatic \glspl{snv} and indels in the 3' \gls{utr} and C-terminus of FOXA1 in \textapprox 12\% of \gls{mcrpc} patients \cite{annalaFrequentMutationFOXA12018}.
In addition to \glspl{snv}, the FOXA1 locus is a target of structural rearrangements in both primary and metastatic \gls{pca} tumours, inclusive of duplications, amplifications, and translocations \cite{paroliaDistinctStructuralClasses2019,adamsFOXA1MutationsAlter2019}.
Taken together, FOXA1 is recurrently mutated taking into account both its coding and flanking noncoding sequences across various stages of \gls{pca} development.

FOXA1 serves as a pioneer \gls{tf} that can bind to heterochromatin, promoting its remodelling to increase accessibility for the recruitment of other \glspl{tf} \cite{yangCurrentPerspectivesFOXA12015}.
FOXA1 binds to chromatin at cell-type specific genomic coordinates facilitated by the presence of mono- and dimethylated lysine 4 of histone H3 (H3K4me1 and H3K4me2) histone modifications \cite{lupienFoxA1TranslatesEpigenetic2008,eeckhouteCelltypeSelectiveChromatin2008}.
In \gls{pca}, FOXA1 is known to pioneer and reprogram the binding of \gls{ar} alongside HOXB13 \cite{pomerantzAndrogenReceptorCistrome2015}.
Independent from its role in \gls{ar} signalling, FOXA1 also regulates the expression of genes involved in cell cycle regulation in \gls{pca} \cite{imamuraFOXA1PromotesTumor2012,imamuraFOXA1PromotesTumor2012,xuAndrogensInduceProstate2006}.
For instance, FOXA1 co-localizes with CREB1 to regulate the transcription of genes involved in cell cycle processes, nuclear division and mitosis in \gls{mcrpc} \cite{imamuraFOXA1PromotesTumor2012,jinAndrogenReceptorIndependentFunction2013,xuAndrogensInduceProstate2006,yangFOXA1PotentiatesLineagespecific2016,zhangFOXA1DefinesCancer2016,augelloFOXA1MasterSteroid2011,sunkelIntegrativeAnalysisIdentifies2017}.
FOXA1 has also been shown to promote feed-forward mechanisms to drive disease progression \cite{niAmplitudeModulationAndrogen2013,sasseFeedforwardTranscriptionalProgramming2015}.
Hence, FOXA1 contributes to \gls{ar}-dependent and \gls{ar}-independent processes favouring \gls{pca} development.

Despite the oncogenic roles of FOXA1, therapeutic avenues to inhibit its activity in \gls{pca} are lacking.
In the breast cancer setting for instance, the use of cyclin-dependent kinases inhibitors have been suggested based on their ability to block FOXA1 activity on chromatin \cite{wangHighThroughputChemical2018}.
As such, understanding the governance of FOXA1 \gls{mrna} expression offers an alternative strategy to find modulators of its activity.
Gene expression relies on the interplay between distal \glspl{cre}, such as enhancers and anchors of chromatin interaction, and their target gene promoter(s) \cite{rowleyOrganizationalPrinciples3D2018}.
These elements can lie tens to hundreds of \glspl{kbp} away from each other on the linear genome but physically engage in close proximity with each other in the three-dimensional space \cite{vernimmenHierarchyTranscriptionalActivation2015}.
By measuring contact frequencies between loci through the use of \gls{3c}-based technologies, it enables the identification of regulatory plexuses corresponding to sets of \glspl{cre} in contact with each other \cite{sallariConvergenceDispersedRegulatory2016,baileyNoncodingSomaticInherited2016}.
By leveraging these technologies, we can begin to understand the three-dimensional organization of the \gls{pca} genome and delineate the FOXA1 regulatory plexus.

Here, we integrate epigenetics and genetics from \gls{pca} patients and model systems to delineate \glspl{cre} establishing the regulatory plexus of FOXA1.
We functionally validate a set of six mutated \glspl{cre} that regulate FOXA1 \gls{mrna} expression.
We further show that \glspl{snv} mapping to these \glspl{cre} are capable of altering their transactivation potential, likely through modulating the binding of key \gls{pca} \glspl{tf}.
