\section{Discussion}

Modern technologies and understanding of the epigenome allow the possibility of probing CRE(s) involved in regulating genes implicated in disease.
Despite FOXA1 being recurrently mutated \cite{abeshouseMolecularTaxonomyPrimary2015,fraserGenomicHallmarksLocalized2017,barbieriExomeSequencingIdentifies2012,grassoMutationalLandscapeLethal2012,robinsonIntegrativeClinicalGenomics2015} and playing potent oncogenic roles in prostate cancer etiology \cite{paroliaDistinctStructuralClasses2019,adamsFOXA1MutationsAlter2019,gaoForkheadDomainMutations2019}, the CREs involved in its transcriptional regulation are poorly understood.
Understanding how FOXA1 is expressed can provide a complementary strategy to antagonize FOXA1 in prostate cancer.

We used the DHS profiled in prostate cancer cells to identify putative FOXA1 CREs through annotating these regions with five different histone modifications, TF binding sites and noncoding SNVs profiled in prostate cancer cells and primary prostate tumours.
Our efforts identified and validated a set of six active CREs involved in FOXA1 regulation, agreeing with a recent report where a subset of our CREs map to loci suggested to be in contact with the FOXA1 promoter \cite{rhieHighresolution3DEpigenomic2019}.
The disruption of these six distal CREs each significantly reduced FOXA1 mRNA levels, similar to what has been demonstrated for ESR1 in luminal breast cancer \cite{baileyNoncodingSomaticInherited2016}, MLH1 in Lynch syndrome \cite{liuDisruption35Kb2018}, MYC in lung adenocarcinoma and endometrial cancer \cite{zhangIdentificationFocallyAmplified2016}, and AR in mCRPC \cite{takedaSomaticallyAcquiredEnhancer2018,viswanathanStructuralAlterationsDriving2018}.
Through combinatorial deletion of two CREs, FOXA1 mRNA levels were further reduced in comparison with single CRE deletions, raising the possibility of CRE additivity \cite{osterwalderEnhancerRedundancyProvides2018}.
The deletion of the FOXA1 plexus CREs also significantly reduced prostate cancer cell proliferation at levels comparable to what has been reported upon deletion of the amplified CRE upstream of the AR gene in mCRPC \cite{takedaSomaticallyAcquiredEnhancer2018}, suggestive of onco-CREs as reported in lung \cite{zhangIdentificationFocallyAmplified2016} and prostate \cite{takedaSomaticallyAcquiredEnhancer2018} cancer.

More than 90\% of SNVs found in cancer map to the noncoding genome \cite{meltonRecurrentSomaticMutations2015,mazrooeiCistromePartitioningReveals2019} with a portion of these SNVs mapping to CREs altering their transactivation potential \cite{baileyNoncodingSomaticInherited2016,zhangIntegrativeFunctionalGenomics2012,huangHighlyRecurrentTERT2013,hornTERTPromoterMutations2013} and/or downstream target gene expression \cite{zhouEmergenceNoncodingCancer2016,meltonRecurrentSomaticMutations2015,weinholdGenomewideAnalysisNoncoding2014}.
We extended this concept with SNVs identified from primary prostate tumours mapping to FOXA1 plexus CREs.
We observed that a subset of these SNVs can alter transactivation potential by modulating the binding of specific TFs whose cistromes are preferentially burdened by SNVs in primary prostate cancer \cite{mazrooeiCistromePartitioningReveals2019}.
Our findings complement recent reports of SNVs found in the noncoding space of FOXA1 that could affect its expression \cite{annalaFrequentMutationFOXA12018,camcapstudygroupSequencingProstateCancers2018}.
The FOXA1 plexus CREs we identified here are also reported to be target of structural variants in both the primary and metastatic settings \cite{paroliaDistinctStructuralClasses2019,quigleyGenomicHallmarksStructural2018}, including tandem duplication in \\textapprox 14\% (14/101) mCRPC tumours over CRE2 \cite{quigleyGenomicHallmarksStructural2018}, amplification, duplication and translocation over CRE3, 4, 5 \cite{paroliaDistinctStructuralClasses2019}.
Notably, the translocation and duplication defining the FOXMIND enhancer driving FOXA1 expression reported in primary and metastatic settings harbors the CRE3 element we characterized \cite{paroliaDistinctStructuralClasses2019}.
Collectively, these studies combined with our discoveries reveal the fundamental contribution of the FOXA1 plexus in prostate cancer etiology.
As a whole, our findings in conjunction with recent reports suggest that CREs involved in the transcriptional regulation of FOXA1 may be hijacked in prostate tumours through various types of genetic alterations.

Despite initial treatment responses from treating aggressive primary and metastatic prostate cancer through castration to suppress AR signalling \cite{attardProstateCancer2016}, resistance ensues as 80\% of mCRPC tumours harbor either AR gene amplification, amplification of a CRE upstream of AR, or activating AR coding mutations \cite{robinsonIntegrativeClinicalGenomics2015,takedaSomaticallyAcquiredEnhancer2018,quigleyGenomicHallmarksStructural2018}.
Given the AR-dependent \cite{yangCurrentPerspectivesFOXA12015,pomerantzAndrogenReceptorCistrome2015} and AR-independent \cite{sunkelIntegrativeAnalysisIdentifies2017} oncogenic activity of FOXA1 in prostate cancer, its inhibition is an appealing alternative therapeutic strategy.
Our dissection of the FOXA1 cis-regulatory landscape complement recent findings through revealing loci that are important for the regulation of FOXA1.
Theoretically, direct targeting of the CREs regulating FOXA1 would down-regulate FOXA1 levels and could therefore serve as a valid alternative to antagonize its function.

Taken together, we identified FOXA1 CREs targeted by SNVs that are capable of altering transactivation potential through the modulation of key prostate cancer TFs.
The study supports the importance of considering CREs not only as lone occurrences but as a team that works together to regulate their target genes, particularly when considering the impact of genetic alterations.
As such, our work builds a bridge between the understanding of FOXA1 transcriptional regulation and new routes to FOXA1 inhibition.
Aligning with recent reports \cite{paroliaDistinctStructuralClasses2019,adamsFOXA1MutationsAlter2019,gaoForkheadDomainMutations2019}, our findings support the oncogenic nature of FOXA1 in prostate cancer.
Gaining insight on the cis-regulatory plexuses of important genes such as FOXA1 in prostate cancer may provide new avenues to inhibit other drivers across various cancer types to halt disease progression.
