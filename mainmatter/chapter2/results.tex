\section{Results}

\subsection{\emph{FOXA1} is essential for prostate cancer proliferation}

We interrogated FOXA1 expression levels across cancer types.
We find that FOXA1 mRNA is consistently the most abundant in prostate tumours compared to 25 other cancer types across patients (fig1a), ranking in the 95th percentile for 492 of 497 prostate tumours profiled in TCGA (figs1a).
Using the same dataset we also find that FOXA1 is the most highly expressed out of 41 other Forkhead Box (FOX) factors in prostate tumours (figs1b).
We next analyzed expression data from DEPMAP and observed FOXA1 to be most highly expressed in prostate cancer cell lines compared to cell lines of other cancer types (figs2a).
Amongst the eight prostate cancer cell lines in the dataset (22Rv1, DU145, LNCaP, MDA-PCa-2B, NCI-H660, PrECLH, PC3, and VCaP), FOXA1 mRNA abundance is above the 90th percentile in all but one cell line (PrECLH) compared to the \> 56,000 protein coding and non-protein coding genes profiled (figs2b).
These new results gained from the TCGA and DEPMAP validate previous understanding that FOXA1 is one of the highest expressed genes in prostate cancer \cite{tsourlakisFOXA1ExpressionStrong2017}.

\newfigure{Fig1.png}{FOXA1 is highly expressed in prostate cancer and essential for prostate cancer cell proliferation.}{\textbf{a}. The mRNA expression of FOXA1 across tumour types ($n = 26$) from RNA-seq data of TCGA. \textbf{b}. FOXA1 essentiality mediated through RNAi across various cell lines ($n=707$) from DEPMAP. Gene essentiality scores are normalized Z-scores. Higher scores indicate less essential, and lower scores indicate more essential for cell proliferation. X-axis indicate tissue of origin for each cell line tested. Each dot indicates one cell line. \textbf{c}. Gene essentiality mediated through RNAi across prostate cancer cell lines ($n=8$) from DEPMAP. Each dot indicates one gene, red indicates FOXA1. \textbf{d}. Representative Western blot against FOXA1 in LNCaP cells 5 days post-transfection of non-targeting siRNA and two independent siRNA targeting FOXA1. \textbf{e}. Cell proliferation assay conducted in LNCaP cells upon siRNA-mediated knockdown of FOXA1 across 5 days. \textbf{f}. Cell proliferation assay conducted in VCaP cells upon siRNA-mediated knockdown of FOXA1 across 5 days. Error bars indicate $\pm$ s.d. $n=3$ independent experiments. Mann-Whitney U test, * $p < 0.05$, ** $p < 0.01$.}{fig:FOXA1_fig1}

Following up on FOXA1 mRNA expression levels, we interrogated the essentiality of FOXA1 for prostate cancer cell growth.
RNAi-mediated essentiality screens compiled in DEPMAP show that FOXA1 lies in the 94th percentile across 6 of the 8 available prostate cancer cell lines: 22Rv1, LNCaP, MDA PCa 2B, NCI-H660, PC3, and VCaP cells (fig1b-c).
The median RNAi-mediated essentiality score for all prostate cell lines is significantly lower than all other cell lines, suggesting that FOXA1 is especially essential for prostate cancer cell proliferation (permutation test, $p = 1 \cdot 10^{-6}$, see Methods) (figs3a).
% Growth assays in LNCaP and VCaP cells following FOXA1 knockdown using two independent siRNAs (fig:FOXA1_fig1d, fig:FOXA1_figs3b) show significant growth inhibition in LNCaP (siRNA \#1: 4-fold, siRNA \#2: 3.35-fold) and VCaP (siRNA \#1: 8.7-fold, siRNA \#2: 2-fold) cells five days post-transfection (Mann-Whitney U Test, $p<0.05$; fig1e-f).
% In accordance with previous reports, our results using essentiality datasets followed by knockdown validation reveals that FOXA1 is oncogenic and essential for prostate cancer cell proliferation.

% \subsection{Identifying putative \emph{FOXA1} CREs}

