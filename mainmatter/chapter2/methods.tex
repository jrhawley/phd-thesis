\section{Methods}
\label{sec:FOXA1_methods}

\subsection{Cell Culture}

LNCaP and 22Rv1 cells were cultured in RPMI medium, and VCaP cells were cultured in DMEM medium, both supplemented with 10\% FBS, and 1\% penicillin-streptomycin at 37 \textdegree C in a humidified incubator with 5\% CO2.
These \gls{pca} cells originated from ATCC.
293FT cells were purchased from ThermoFisherScientific (Cat No. R70007) maintained in complete DMEM medium (DMEM with 10\% FBS (080150, Wisent), L-glutamine ($25030-081$, ThermoFisher) and non-essential amino acids ($11140-050$, ThermoFisher) supplemented with 50 mg/mL Geneticin (4727894001, Sigma-Aldrich).
The cells are regularly tested for \emph{Mycoplasma} contamination.
The authenticity of these cells was confirmed through \gls{str} profiling.

\subsection{Prostate tumours and cancer cell lines expression}

Cancer cell line \gls{mrna} abundance data were collected from \gls{depmap} (\url{https://depmap.org/portal/}; \gls{rnaseq} TPM values from 2018q4 version with all 5 non-cancer cell lines were removed) \cite{thecancercelllineencyclopediaconsortiumPharmacogenomicAgreementTwo2015} projects.
Prostate tumour \gls{mrna} abundance data was collected from \gls{tcga} \glsentrylong{pca} (\gls{tcga}-PRAD) project via the Xena Browser (\url{https://xenabrowser.net/}; dataset description: \gls{tcga} prostate adenocarcinoma gene expression by \gls{rnaseq} (polyA+ Illumina HiSeq; RSEM)).

\subsection{Prostate cancer cell line gene essentiality}

Essentiality scores were collected from the \gls{depmap} Project \cite{mcfarlandImprovedEstimationCancer2018}.
To compare gene essentiality between \gls{pca} cell lines and others, essentiality scores for \emph{FOXA1} were collected from all available cell lines ($n = 707$).
To perform a permutation test, the median expression of 8 randomly selected cell lines was calculated one million times to generate a background distribution of essentiality scores across all cell types available.
The median essentiality score from the 8 \gls{pca} cell lines was calculated and its percentile within the background distribution is reported.

\subsection{\glsentryshort{sirna} knockdown and cell proliferation assay}

300,000 LNCaP cells (Day 0) were reverse transfected with \gls{sirna} (siFOXA1 using Lipofectamine \gls{mrna}max reagent; ThermoFisher Scientific, Cat No. 13778150).
Cells were counted using Countess automated cell counter (Invitrogen).
Whole cell lysates LNCaP cells after \gls{sirna}-mediated \emph{FOXA1} knockdown was collected at 96 h post-transfection in RIPA buffer.
Protein concentrations were determined through the bicinchoninic acid method (ThermoFisher Scientific, Cat No. 23225).
Then 25 $\mu$g of lysate was subjected to \gls{sds}-PAGE.
Upon completion of \gls{sds}-PAGE, protein was transferred onto PVDF membrane (Bio-Rad, Cat No. 1704156).
The membrane was blocked with 5\% non-fat milk for 1 h at room temperature with shaking.
After blocking, anti-\emph{FOXA1} (Abcam Cat No. 23737) in 2.5\% non-fat milk was added, and was incubated at 4 \textdegree C overnight.
Next day, the blot was washed and incubated with IRDye 800CW Goat Anti-Rabbit IgG secondary antibody (LI-COR, Cat No. $925-32211$) at room temperature for 1 h.
The blot was then washed and assessed with the Odyssey CLX imaging system (LI-COR).

\subsection{Identifying putative \emph{FOXA1} \glsentryshortpl{cre}}

Putative \emph{FOXA1} \glspl{cre} were identified through the use of C3D method based on DNase I Hypersensitivity \cite{mehdiC3DToolPredict2019}.
Predicted interacting \gls{dhs} with a Pearson's correlation above 0.7 \cite{thurmanAccessibleChromatinLandscape2012} were kept for downstream analysis.

\subsection{Hi-C and \glsentryshortpl{tad} in LNCaP cells}

Hi-C and \glspl{tad} conducted and called, respectively, in LNCaP cells are publicly available off ENCODE portal (experiment accession ENCSR346DCU; FASTQ file accessions: ENCFF726LGW, ENCFF550SKU, ENCFF950CQX, and ENCFF411TZJ; \glspl{tad} file accession: ENCFF139JCA).
Visualization of the Hi-C dataset is available on the Hi-C Browser \cite{wang3DGenomeBrowser2018}.

\subsection{Clonal \glsentrylong{wt} Cas9 and dCas9-KRAB mediated validation}

Lentiviral particles were generated in 293FT cells (ThermoFisher) using the pMDG.2 and psPAX2 packaging plasmids (Addgene; \#12259 and \#12260, a gift from Didier Trono) alongside the Lenti-Cas9-2A-Blast plasmid (Addgene \#73310, a gift from Jason Moffat) and collected 72 hrs post transfection.
LNCaP and 22Rv1 cells were then transduced for $24-48$ h with equal amounts of virus followed by selection with media containing blasticidin (7.5 $\mu$g/mL for LNCaP cells, 6 $\mu$g/mL for 22Rv1 cells).
Upon selection, clones were derived by serial dilution with subsequent single cell seeding into 96-well plates containing selection media.
Cas9 protein expression for each clone was then assessed through Western blotting (primary Ms-Cas9 (Cell Signalling Technology, Cat No. \#14697) 1:1000, Ms-GAPDH 1:5000 (Santa Cruz Biotechnology, Cat No. \#sc47724) in 5\% non-fat milk; secondary HRP-linked Anti-Mouse IgG (Cell Signalling Technology, Cat No. \#7076S) 1:10 000 in 2.5\% non-fat milk.
The full unprocessed blot is in the Source Data File.

Lentiviral particles were generated in 293FT cells (ThermoFisher) using the pMDG.2 and psPAX2 packaging plasmids (Addgene; \#12259 and \#12260, a gift from Didier Trono) alongside the Lenti-dCas9-KRAB-blast plasmid (Addgene \#89567, a gift from Gary Hon) and collected 72 hrs post transfection.
LNCaP and 22Rv1 cells were then transduced for $24-48$ h with equal amounts of virus followed by selection with media containing blasticidin (7.5 $\mu$g/mL for LNCaP cells, 6 $\mu$g/mL for 22Rv1 cells).
Upon selection, clones were derived by serial dilution with subsequent single cell seeding into 96-well plates containing selection media.
dCas9-KRAB protein expression for each clone was then assessed through Western blotting (1 \textdegree Ms-Cas9 (Cell Signalling Technology, Cat No. \#14697) 1:1000, Ms-GAPDH 1:5000 (Santa Cruz Biotechnology, Cat No. \#sc47724) in 5\% non-fat milk; 2 \textdegree HRP-linked Anti-Mouse IgG (Cell Signalling Technology, Cat No. \#7076S) 1:10 000 in 2.5\% non-fat milk.
The full unprocessed blot is in the Source Data File.

For \gls{grna} design, five to six unique \gls{crrna} molecules (Integrated DNA Technologies) were designed to tile across the region of interest using the CRISPOR tool (\url{http://crispor.tefor.net/}) \cite{haeusslerEvaluationOfftargetOntarget2016} and the Zhang lab CRISPR Design tools (\url{http://crispr.mit.edu/}) \cite{hsuDNATargetingSpecificity2013}.
See published manuscript for \gls{grna} sequences.
Each \gls{crrna} and \gls{tracrrna} (Integrated DNA Technologies) were duplexed according to company supplier protocol to a concentration of 50 $\mu$M.
Upon generation of the clones, six guides (\gls{crrna}-\gls{tracrrna} duplexes) for each region of interest were pooled into a single tube (1 $\mu$L each guide, 6 $\mu$L per reaction) (Integrated DNA Technologies).
Lastly, 1 $\mu$L (100 $\mu$M) of electroporation enhancer (Integrated DNA Technologies) was added to the mix (7 $\mu$L total) prior to transfection.
The entire transfection reaction was transfected into 350,000 cells through Nucleofection (SF Solution EN120 - 4D Nucleofector, Lonza).
Cells were then harvested 24 h post-transfection for RNA and DNA for RT-PCR and confirmation of deletion, respectively.

\subsection{Transient Cas9-mediated disruption of \glsentryshortpl{cre}}

Deletion of elements through this method were achieved through the transfection of Cas9 nuclease protein complexed with the \gls{crrna} (Integrated DNA Technologies).
Briefly, five to six unique \gls{crrna} molecules (Integrated DNA Technologies) were designed to tile across the region of interest using the CRISPOR tool (\url{http://crispor.tefor.net/}) \cite{haeusslerEvaluationOfftargetOntarget2016} and the Zhang lab CRISPR Design tools (\url{http://crispr.mit.edu/}) \cite{hsuDNATargetingSpecificity2013}.
Each \gls{crrna} and \gls{tracrrna} (Integrated DNA Technologies) were duplexed according to company supplier protocol to a concentration of 50 $\mu$M.
The six \gls{crrna}-\gls{tracrrna} duplexes were pooled into a single tube (6 $\mu$L per reaction), prior to adding 1 $\mu$L (5 $\mu$g) of Alt-R S.p. HiFi Cas9 Nuclease 3NLS (Integrated DNA Technologies).
The reaction was incubated at room temperature for 10 min for ribonucleoprotein (RNP) complex formation.
Lastly, 1 $\mu$L (100 $\mu$M) of electroporation enhancer (Integrated DNA Technologies) was added to the mix prior to transfection.
The entire transfection reaction was transfected into 350 000 cells through Nucleofection (SF Solution EN120 - 4D Nucleofector, Lonza).
Cells were then harvested 24 h post-transfection for RNA and DNA for RT-PCR and confirmation of deletion, respectively.
For double deletions, two sets of \gls{grna}-RNP complex (10 $\mu$g of Alt-R S.p. HiFi Cas9 Nuclease 3NLS) were transfected and harvested 24 h post-transfection for RNA and DNA for RT-PCR and confirmation of deletion, respectively.
To control for double deletions, two negative control regions within the \gls{tad} were also compounded.
%% Fix this
Due to size, see published manuscript for primers.

\subsection{RT-PCR assessment of gene expression upon deletion of \glsentryshortpl{cre}}

DNA and RNA were harvested with Qiagen AllPrep RNA/DNA Kit (Qiagen, Cat No. 80204).
Next, cDNA was synthesized from 300 ng of RNA using SensiFast cDNA Synthesis kit (Bioline, Cat No. BIO-65054), and \gls{mrna} expression levels for various genes of interest were assessed.
Due to size, see published manuscript for the primer sequences used for expression evaluation.
Differential gene expression was calculated by normalizing against \emph{TBP} (housekeeping gene).
Statistical significance was calculated using Student's $t$-test in R.

\subsection{Confirmation of Cas9-mediated deletion of \glsentryshortpl{cre}}

Deletion of \glspl{cre} were confirmed through PCR amplification of the intended region for deletion, followed by the T7 Endonuclease Assay (Integrated DNA Technology).
%% fix this
Due to size, see published manuscript for primer sequences used for PCR amplification.
PCR products were then loaded onto a 1\% agarose gel for visualization.
The agarose gel to assess the on-target genome editing efficiency was done through densitometry using ImageJ.
The correlation between on-target genome editing efficiency and \emph{FOXA1} \gls{mrna} expression reduction was drawn through Pearson's correlation in R.

\subsection{Cell proliferation upon deletion of \emph{FOXA1} \glsentryshortpl{cre}}

Pairs of \glspl{grna} flanking the \glspl{cre} of interest, \emph{FOXA1} promoter and control regions were designed using CRISPOR (\url{http://crispor.tefor.net/}) and Zhang lab CRISPR Design tool (\url{http://crispr.mit.edu/}) (due to size, see published manuscript).
Each pair of \glspl{grna} were cloned into the lentiCRISPRv2 (Addgene; a gift from Feng Zhang \#52961) and the lentiCRISPRv2-Blast (Addgene; a gift from Feng Zhang \#83480) plasmid as previously described \cite{sanjanaImprovedVectorsGenomewide2014}.
Lentiviral particles were generated in 293FT cells (ThermoFisher) using the pMDG.2 and psPAX2 packaging plasmids (Addgene; \#12259 and \#12260, a gift from Didier Trono), and collected 72 hrs post transfection.
LNCaP cells were transduced for $24-48$ h with equal amounts of virus, followed by selection with media containing puromycin (3.5 $\mu$g/mL, ThermoFisher) and blasticidin (7 $\mu$g/mL, Wisent).
Cells were harvested upon selection for RNA and DNA for RT-PCR and confirmation of DNA cleavage, respectively.
For cell proliferation, cells were seeded at equal density per well (on a 96-well plate; Day 1) upon puromycin and blasticidin selection.
Growth of the cells were monitored through cell counting using Countess automated cell counter (Invitrogen).
Cell numbers were calculated as a percentage compared to negative control.
Statistical significance was calculated using Student's $t$-test.

\subsection{Luciferase reporter assays}

Each region of interest was ordered as gBlocks from Integrated DNA Technologies.
The regions were cloned into the BamHI restriction enzyme digest site of the pGL3 promoter plasmid (Promega).
On Day 0, 90 000 LNCaP cells were seeded in 24-well plates.
Next day (Day 1), pGL3 plasmids harboring the \gls{wt} and variant sequences were co-transfected with the pRL Renilla plasmid (Promega) using Lipofectamine 2000.
48 h later, the cells were harvested, and dual luciferase reporter assays were conducted (Promega).
Notably, inserts of both forward and reverse directions were tested using this assay as enhancer elements are known to be direction-independent.
Final luminescence readings are reported as firefly luciferase normalized to renilla luciferase activity.
The assessment of each mutation was conducted in five biological replicates.
Statistical significance was assessed by Mann-Whitney U test in R.
See published manuscript for gBlock sequences.

\subsection{Allele-specific ChIP-qPCR}

Briefly, pGL3 plasmids containing the \gls{wt} sequence and the mutant sequence used in the luciferase reporter assay were transfected into 7 million cells (2 $\mu$g per allele, per 1 million cells) using Lipofectamine 2000 (ThermoFisher Scientific), per manufacturer's instructions.
Next day, each antibody (\emph{FOXA1} 5 $\mu$g, Abcam, ab23738; \gls{ar} 5 $\mu$g, Abcam, ab1083241; HOXB13 5 $\mu$g, Abcam, ab201682; SOX9 5 $\mu$g, Abcam, ab3697; GATA2 5 $\mu$g, Abcam, ab22849; FOXP1 5 $\mu$g, Abcam, ab16645; NKX3.1 10 $\mu$l, Cell Signalling Technology, \#83700) was conjugated with 10 $\mu$L of each Dynabeads A and G (Thermo Fisher Scientific) for each ChIP for 6 h with rotation at 4 \textdegree C.
When antibody-beads conjugates were ready for use, cells were lifted using trypsin and fixed by re-suspending with 300 $\mu$L of 1\% formaldehyde in \gls{pbs} for 10 min at room temperature.
2.5 M Glycine was added to quench excess formaldehyde (final concentration 0.125 M).
Cells were then washed with cold \gls{pbs} and lysed using 300 $\mu$L of Modified RIPA buffer (10 mM Tris-HCl, pH 8.0; 1 mM EDTA; 140 mM NaCl; 1\% Triton X-100; 0.1\% \gls{sds}; 0.1\% sodium deoxycholate) supplemented with protease inhibitor.
The lysate was subject to 25 cycles of sonication (30s ON 30s OFF) using Diagenode Bioruptor Pico (Diagenode).
15 $\mu$L of sonicated lysate was set aside as input with the rest used for chromatin pulldown through addition of antibody-beads conjugates and overnight incubation at 4 \textdegree C with rotation.
Next day, the beads were washed once with Modified RIPA buffer, washed once with Modified RIPA buffer + 500 mM NaCl, once with LiCl buffer (10 mM TrisHCl, pH 8.0; 1 mM EDTA; 250 mM LiCl; 0.5\% NP-40; 0.5\% sodium deoxycholate) and twice with Tris-ETDA buffer (pH 8).
After washes, beads and input were de-crosslinked by addition of 100 $\mu$L de-crosslinking buffer and incubation at 65 \textdegree C for 6 h.
Samples were then purified and eluted.
ChIP and input DNA were then used for allele-specific ChIP-qPCR using MAMA primers as described previously.
fold change significance was calculated using Student's $t$-test in R.

All analyses were done using GRCh37 \cite{internationalhumangenomesequencingconsortiumFinishingEuchromaticSequence2004} reference genome coordinates.
