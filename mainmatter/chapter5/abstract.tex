\section{Abstract}

Relapse of \glsentrylong{ball} remains a significant cause of death in treating the disease.
Genomic investigations indicate that relapsed disease often arises from a minor clone of cells present at diagnosis.
However, both genetic and epigenetic variation have been observed in \glsentrylong{ball} and other leukemias, and why come cells with particular genetic or epigenetic profiles survive therapy remains unknown.
Here, we use targeted genome sequencing, \glsentrylong{rnaseq}, \glsentrylong{atacseq}, and bisulfite sequencing of patient-matched samples with \glsentrylongpl{pdx} to investigate the dynamics of the genome and epigenome over \glsentrylong{ball} relapse.
We find that \glsentrylong{dname} profiles most closely resemble genetic clones at diagnosis and relapse.
Moreover, we find widespread increases to \glsentrylong{dname} at relapse, mirroring a more stem-like phenotype.
This work suggests that therapy selects for clones with stem-like characteristics, both genetically and epigenetically in \glsentrylong{ball}.
