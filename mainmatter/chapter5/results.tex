\section{Results}

\subsection{Multiomic integration of \glsentryshort{ball} relapse patients links \glsentrylong{dname} to relapse status}

\newfigure{chapter5/Figure1.png}{Experimental design and data integration}{\textbf{a.} Experimental design of samples used in this study. Normal samples were obtained from cord blood pools and sorted into various hematopoietic cell types. B-ALL patients who experienced relapse has sorted leukemic blasts collected at \gls{dx} and \gls{rel}. Based on the mutation profiles from \cite[REF][]{dobsonRelapseFatedLatentDiagnosis2020} some \gls{dx} samples are labelled as \gls{dri}. \textbf{b.} Individual and fused networks of samples from three patients with complete multiomic profiling. Nodes represent individual samples (either primary or \gls{pdx}), edges represent similarities between the connected samples.}{fig:BALL_fig1}

\subsection{Widespread loss of \glsentrylong{dname} over normal B-cell differentiation}

\newfigure{chapter5/Figure2.png}{Loss of \Glsentrylong{dname} over B-cell differentiation}{\textbf{a.} Schematic of the hematopoietic hierarchy and the grouping of B-cell progenitors into the groups isolated in this study. \textbf{b.} Heatmap of \glspl{dmr} identified between B lineage cell types. Columns are samples ordered by cell type and rows are \glspl{dmr} identified in at least one pairwise comparison between cell types (dmrseq, \gls{fdr} $< 0.1$). \textbf{c.} Bar plot of \glspl{dmr} classified by which step in differentiation they were identified as significantly changed.}{fig:BALL_fig2}

\subsection{Widespread gain of \glsentrylong{dname} over \glsentryshort{ball} relapse}

\newfigure{chapter5/Figure3.png}{Gain of \glsentrylong{dname} over \gls{ball} relapse}{\textbf{a.} Heatmaps of \glspl{dmr} identified between \gls{dx} and \gls{rel} samples within each patient. \textbf{b.} Upset plot showing the shared \glspl{dmr} between patients. \Glspl{dmr} in the left highlighted block are unique to a single patient, whereas \glspl{dmr} in the right highlighted block are recurrent changes across all 5 relapse patients. These \glspl{dmr} are in the promoter regions of the callout genes listed.}{fig:BALL_fig3}

\subsection{Recurrent \glsentrylong{dname} changes identify stem cell pathways in relapse}

\newfigure{chapter5/Figure4.png}{Recurrently hypermethylation gene promoters are associated with cell fate decision processes}{\textbf{a.} \Gls{go} analysis of genes with recurrently hypermethylated promoters in \gls{rel} \gls{ball} samples. The red dashed line indicates the \gls{fdr} threshold of 0.05. \textbf{b.} Pairwise \gls{dname} changes in each patient at the recurrently hypermethylated loci show increased methylation in all patients.}{fig:BALL_fig4}
