\section{Results}

\subsection{Multiomic integration of \glsentryshort{ball} relapse patients links \glsentrylong{dname} to relapse status}

\begin{itemize}
  \item We profiled primary and \gls{pdx} \gls{ball} samples using \gls{rnaseq}, \gls{atacseq}, and \gls{mecapseq} \Cref{fig:BALL_fig1}a
  \item To contextualize any changes we may find in \gls{ball} relapse, we first looked to the hematopoietic hierarchy to characterize \gls{dname} changes that occur over the course of B-cell differentiation
  \item To identify which molecular marker was most indicative of disease stage, we integrated these three data types with disease stage classifications from \cite[REF][]{dobinSTARUltrafastUniversal2013}
  \item We then performed \gls{snf} to construct clusters of samples from all three molecular datasets, both combined and individually
  \item We found for the three patients whose multiomic data passed quality control checks for all datasets, that \gls{dname} was the individual dataset that most strongly correlated with disease state, behind the combined graphs \Cref{fig:BALL_fig1}b
\end{itemize}

\newfigure{chapter5/Figure1.png}{Experimental design and data integration}{\textbf{a.} Experimental design of samples used in this study. Normal samples were obtained from cord blood pools and sorted into various hematopoietic cell types. B-ALL patients who experienced relapse has sorted leukemic blasts collected at \gls{dx} and \gls{rel}. Based on the mutation profiles from \cite[REF][]{dobsonRelapseFatedLatentDiagnosis2020} some \gls{dx} samples are labelled as \gls{dri}. \textbf{b.} Individual and fused networks of samples from three patients with complete multiomic profiling. Nodes represent individual samples (either primary or \gls{pdx}), edges represent similarities between the connected samples.}{fig:BALL_fig1}

\subsection{Widespread loss of \glsentrylong{dname} over normal B-cell differentiation}

\begin{itemize}
  \item Given the strong correlation between \gls{dname} signal and disease state, we decided to investigate \gls{dname} changes in \gls{ball} relapse
  \item We performed \gls{mecapseq} on 8 normal cord blood pools, separating into various cell types based on cell surface markers \Cref{tab:BALL_markers}
  \item This identified 540 \glspl{dmr}, 500 (92.6 \%) of which become hypomethylated over the course of differentiation
  \item No \glspl{dmr} change methylation levels then revert back; once a region changes its methylation, that change persists throughout differentiation
  \item These findings are corroborated by previous studies of B-cell differentiation using the Illumina 450K array \cite{leeGlobalDNAMethylation2012,leeEpigeneticRemodelingBcell2015,nordlundGenomewideSignaturesDifferential2013}
  \item In summary, normal hematopoietic stem cells permanently change \gls{dname} over the course of differentiation, predominantly by losing \gls{dname}
\end{itemize}

\newfigure{chapter5/Figure2.png}{Loss of \Glsentrylong{dname} over B-cell differentiation}{\textbf{a.} Schematic of the hematopoietic hierarchy and the grouping of B-cell progenitors into the groups isolated in this study. \textbf{b.} Heatmap of \glspl{dmr} identified between B lineage cell types. Columns are samples ordered by cell type and rows are \glspl{dmr} identified in at least one pairwise comparison between cell types (dmrseq, \gls{fdr} $< 0.1$). \textbf{c.} Bar plot of \glspl{dmr} classified by which step in differentiation they were identified as significantly changed.}{fig:BALL_fig2}

\subsection{Recurrent \glsentrylong{dname} changes identify stem cell pathways in relapse}

\begin{itemize}
  \item We searched for \glspl{dmr} that are present at \gls{ball} relapse using the using primary and \gls{pdx} \gls{dx} and \gls{rel} \gls{ball} samples
  \item When grouping all patients together by disease stage, we found no \glspl{dmr} survived multiple testing corrections
  \item We reanalyzed the same data in a patient-oriented approach, identifying \glspl{dmr} that come from each patient's relapse trajectory
  \item We were able to identify ~ 26 000 \glspl{dmr} across the cohort of 5 patients \Cref{fig:BALL_fig3}a
  \item Unlike normal differentiation, most \glspl{dmr} became hypermethylated at relapse
  \item In nearly all cases, the \gls{dx} \gls{pdx} samples strongly resembled the \gls{dx} primary samples, but one \gls{dx} \gls{pdx} sample for Patient 9 more strongly resembles the methylation profile at \gls{rel} \Cref{fig:BALL_fig3}a
  \item This suggests that, like subpopulations identified via mutation profiles, \gls{dname} of a subpopulation of cells present at diagnosis may give rise to the relapse population
  \item Combining the patient-oriented \glspl{dmr} demonstrates that most \glspl{dmr} are patient-specific (\Cref{fig:BALL_fig3}b, left side)
  \item Only a small number of \glspl{dmr} are shared across patients, which are all within promoter regions of the genes highlighted (\Cref{fig:BALL_fig3}b)
  \item Taken together, we find that the changes to \gls{dname} over the course of \gls{ball} relapse is antithetical to the changes seen over normal B-cell differentiation
\end{itemize}

\begin{itemize}
  \item We investigated the potential effects of these recurrent changes to \gls{dname} using \gls{go} analysis
  \item We found numerous biological processes positively associated with differentiation, the most significant of which is cell fate determination (\Cref{fig:BALL_fig3}c)
  \item For the recurrent \glspl{dmr}, all patients had $> 20$ \% gain in methylation in the promoter regions of these genes ()\Cref{fig:BALL_fig3}d)
  \item There are some regions where both hyper- and hypomethylation was observed at \gls{rel}, but all promoters had increased at least some \gls{dmr} with increased methylation (\Cref{fig:BALL_fig3}d)
  \item Taken together, this suggests that the \gls{dname} changes observed at \gls{rel} revert to a more de-differentiated, stem-like state
\end{itemize}

\newfigure{chapter5/Figure3.png}{Recurrent relapse \glspl{dmr} are associated with cell fate decision processes}{\textbf{a.} Heatmaps of \glspl{dmr} identified between \gls{dx} and \gls{rel} samples within each patient. \textbf{b.} Upset plot showing the shared \glspl{dmr} between patients. \Glspl{dmr} in the left highlighted block are unique to a single patient, whereas \glspl{dmr} in the right highlighted block are recurrent changes across all 5 relapse patients. These \glspl{dmr} are in the promoter regions of the callout genes listed. \textbf{c.} \Gls{go} analysis of genes with recurrently hypermethylated promoters in \gls{rel} \gls{ball} samples. The red dashed line indicates the \gls{fdr} threshold of 0.05. \textbf{d.} Pairwise \gls{dname} changes in each patient at the recurrently hypermethylated loci show increased methylation in all patients.}{fig:BALL_fig3}
