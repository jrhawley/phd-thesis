\section{Discussion}

Disease relapse remains a major barrier in treating \gls{ball} \cite{formanMythSecondRemission2013,liewOutcomesAdultPatients2012,hungerAcuteLymphoblasticLeukemia2015}.
While the genetic origins of relapse have been characterized, epigenetic aberrations underlying relapse have been less well-studied.
In this work, we investigated the epigenetic and transcriptomic changes of 5 \gls{ball} patients over the course of relapse to identify non-genetic changes in tumours that may lead to relapse.
\gls{dname} is more highly correlated with disease stage than RNA or chromatin accessibility and changes to \gls{dname} are antithetical to \gls{dname} changes seen in normal B-cell differentiation.
While most \gls{dname} changes are patient-specific, a small number of recurrent changes indicate a more stem-like state at relapse.
In some cases, these stem-like \gls{dname} profiles are present at diagnosis, indicating that subclones defined by \gls{dname} may also contribute to \gls{ball} relapse.

Both genetic and epigenetic aberrations in tumours play important roles in determining disease relapse \cite{nordlundGenomewideSignaturesDifferential2013,leeEpigeneticRemodelingBcell2015}.
Previous reports highlight the frequency that epigenetic regulators are mutated in \gls{ball} \cite{maRiseFallSubclones2015} and leukemias more generally, such as \emph{DNMT3A}, \emph{TET2}, \emph{IDH1}, and \emph{IDH2} in \gls{aml} \cite{kishtagariDriverMutationsAcute2020,papaemmanuilGenomicClassificationPrognosis2016,leyDNMT3AMutationsAcute2010} and \emph{CHD2}, \emph{HIST1H1E}, and \emph{ZMYM3} in \gls{cll} \cite{billotDeregulationAiolosExpression2011,landauChronicLymphocyticLeukemia2013,landauEvolutionImpactSubclonal2013}.
These findings demonstrate that epigenetic modifications, in conjunction with genetic aberrations, discriminate disease outcomes and can share an evolutionary trajectory in cancer.
However, it remains unclear why \gls{dname} changes in \gls{ball} patients remain mostly patient-specific.
One possibility is that the \gls{dname} profile is linked to the genetic profile of the tumour and that this genetic predisposition influences how \gls{dname} changes.
Each of the five patients here harbour different genetic mutations, defining different subtypes of \gls{ball}.
While some genetic subtypes are shared between patients (e.g. Patients 1 and 7 both belong to the DUX4 subtype and share $> 4000$ \glspl{dmr}), the lack of large sample sizes with a common genetic subtype may confound this relationship.
Another possibility is that the selective pressure on cells caused by therapy induces divergent \gls{dname} patterns in a similar fashion to increased mutation rates in cancers after treatment \cite{russoAdaptiveMutabilityColorectal2019}.
Stochastic exploration of fitness landscapes through \gls{dname} changes may lead to the type of \glspl{dmr} observed here; a select few \glspl{dmr} converge on similar biological pathways to evade therapy surrounded by hundreds or thousands of passenger \glspl{dmr} that have no effect.
Distinguishing between these processes would require genetically identical models to separate the effect of genetic profiles on epigenetic dynamics.

However, it is not the case that genetic and epigenetic states always behave similarly.
In this study we found both \gls{dx} and \gls{dri} \glspl{pdx} samples that share \gls{dname} profiles with the \gls{rel} tumours, suggesting that \gls{dname} states can vary independently of mutations.
Moreover, the differences between \gls{pdx} methylomes derived from the same primary sample demonstrates that subpopulations of cells can have differing \gls{dname} states while sharing mutations.
This decoupling between genome and epigenome has been observed in other tumours, such as pediatric ependymomas, where recurrent \gls{dname} profiles were found in the absence of recurrent mutations and was associated with outcome \cite{mackEpigenomicAlterationsDefine2014,pajtlerMolecularClassificationEpendymal2015}, and glioblastoma, where stem cells are characterized by widespread changes in chromatin accessibility \cite{guilhamonSinglecellChromatinAccessibility2021} and histone modifications \cite{liauAdaptiveChromatinRemodeling2017}.
These studies highlight the role of epigenetic plasticity and intra-tumour heterogeneity in cancers \cite{flavahanEpigeneticPlasticityHallmarks2017}.
With similar results found in leukemias that are linked to disease outcome \cite{pastoreCorruptedCoordinationEpigenetic2019,landauLocallyDisorderedMethylation2014,gaitiEpigeneticEvolutionLineage2019,namIntegratingGeneticNongenetic2021,liDistinctEvolutionDynamics2016}, it is likely that epigenetic plasticity and heterogeneity are also key factors in therapeutic response and relapse.
Taken together, these results suggest that the epigenome can provide mechanisms independent of genetic aberrations, to respond and adapt to therapies, but are often guided by genetic aberrations.
This complexity of disease response will need to be addressed to design treatment regimens for patients with an increased propensity towards relapse.

Previous investigations of \gls{dname} aberrations in \gls{ball} have primarily focused on a select few genes, or single \glspl{cpg} in promoter regions \cite{nordlundGenomewideSignaturesDifferential2013,leeEpigeneticRemodelingBcell2015,garcia-maneroDNAMethylationMultiple2002,garcia-maneroAberrantDNAMethylation2003}.
While the recurrent \glspl{dmr} in this study were found in these same regions, most \glspl{dmr} were identified in intergenic regions.
This suggests that important changes in the epigenetic landscape is currently unidentified, and future studies investigating \gls{dname} aberrations in \gls{ball} should prioritize genome-wide approaches.
The phenotypic impact of focal hypermethylation on engraftment and self-renewal capacity has not been assessed here, so experiments should be conducted to validate these findings (this is a bad sentence but this idea is important).
For patients undergoing \gls{ball} treatment, \gls{dname} has the potential to be used as early indicators of relapse.
Moreover, treatment with DNA demethylating agents, such as 5-aza-citidine and 5-aza-2'-deoxycytadine, may be effective at preventing relapse.
These treatments have been approved for use in patients with \gls{mds} and \gls{aml} in adult populations and early clinical trials have demonstrated their safety \cite{bentonSafetyClinicalActivity2014,nationalcancerinstitutenciGroupwidePilotStudy2021}, although some toxic effects have been identified in drug combination trials \cite{therapeuticadvancesinchildhoodleukemiaconsortiumPilotStudyDecitabine2020}.
Taken together, therapeutic targeting of \gls{dname} may be an effective method to prevent \gls{ball} relapse by preventing the outgrowth of stem-like subpopulations that survive chemotherapy.
