\section{Methods}

\subsection{Patient selection and sample collection}

Patient samples were obtained at diagnosis and relapse from patients with \gls{ball} as previously described \cite{dobsonRelapseFatedLatentDiagnosis2020}.
All samples were frozen viably and stored long term at -150 \textdegree C.
Samples were selected retrospectively based on paired-sample availability.

Human cord blood samples were obtained with informed consent from Trillium and Credit Valley Hospital according to procedures approved by the University Health Network Research Ethics Board, as previously described \cite{dobsonRelapseFatedLatentDiagnosis2020}.
Cells were stained with the following antibodies (all from BD Biosciences, unless otherwise stated):

\begin{itemize}
  \item FITC anti-CD45RA (1:50, 555488)
  \item PE anti-CD90 (1:50, 555596)
  \item PE-Cy5 anti-CD49f (1:50, 551129)
  \item V450 anti-CD7 (1:33.3, 642916)
  \item PE-Cy7 anti-CD38 (1:100, 335790)
  \item APC anti-CD10 (1:50, 340923)
  \item APC-Cy7 anti-CD34 (1:200, custom made by BD Biosciences)
\end{itemize}

Cells were sorted from cord blood cells on the basis of markers listed in \Cref{tab:BALL_markers}, as previously described \cite{nottaIsolationSingleHuman2011}, on a FACSAria III (Becton Dickinson), consistently yielding $> 95$ \% purity.

\newlongtable{l | l}{Cell surface markers used to isolate cell populations from cord blood pools}{
  \textbf{Cell type(s)} & \textbf{Surface markers} \\
  \hline
  \glspl{hsc} \& \glspl{mpp} & CD34+ CD38- CD45RA- \\
  \glspl{cmp}, \glspl{gmp}, \& \glspl{mep} & CD34+ CD38+ CD10- CD19+ \\
  \glspl{lmpp} \& \glspl{mlp} & CD34+ CD38- CD45RA+ \\
  \glspl{eprob}, \glspl{preprob}, \& \glspl{prob} & CD34+ CD38+ CD10+ CD19+ \\
  B & CD34- CD38+ CD19+ CD33- CD3- CD56-
}{tab:BALL_markers}{}

\subsection{\Glsentrylong{pdx} generation and \glsentrylongpl{lda}}

\Glspl{pdx} were generated as previously described \cite{dobsonRelapseFatedLatentDiagnosis2020}.
Clinical samples were stained with the following antibodies:

\begin{itemize}
  \item anti-CD19 PE (BD Biosciences, clone 4G7)
  \item anti-CD3 FITC (BS Biosciences, clone SK7) or anti-CD3 APC (Beckman Coulter, clone UCHT11)
  \item anti-CD45 APC (BD Biosciences, clone 2D1) or anti-CD45 FITC (BD Biosciences, clone 2D1)
  \item anti-CD34 APC-Cy7 (BD Biosciences, clone 581)
\end{itemize}

Each sample was sorted on a FACSAria III (BD Biosciences) for leukemic blasts ($\text{CD19}^+ \text{CD45}^{\text{dim/-}}$) and T cells ($\text{CD3}^+ \text{CD45}^{\text{hi}}$).
\Gls{nsg} mice were bred according to protocols established and approved by the Animal Care Committee at the University Health Network.
8-to-12-week-old mice were sublethally irradiated at 225 cGy 24 hours prior to transplants.
Only female mice were used.
Intra-femoral injections of 10 to 250 000 sorted leukemic blasts were performed as previously described \cite{mazurierRapidMyeloerythroidRepopulation2003}.
Mice were sacrificed 20-to-30 weeks post-transplant or at the onset of disease symptoms.
Human cell engraftment in the injected femur, bone marrow (non-injected bones, left tibia, right tibia, left femur), spleen, and central nervous system were assessed using human-specific antibodies for CD45 (PE-Cy7, BD Biosciences, clone HI30; v500 BD Biosciences, clone HI30), CD44 (PE, BD Biosciences, clone 515; FITC, BD Biosciences, clone L178), CD3 (APC, BD Biosciences, clone UCHT1), CD19 (PE-Cy5, Beckman Coulter, clone J3-119), CD33 (PE-Cy7, BD Biosciences, clone P67-6; APC, BD Biosciences, clone P67-6), and CD34 (APC-Cy7, BD Biosciences, clone 581) analyzed on an LSRII (BD Biosciences).
Mice were considered to be engrafted when $> 0.1$ \% of cells in the injected femur were positive for one or more human \gls{ball}-specific cell surface marker (CD45, CD44, CD19, and CD34).
Confidence intervals for the frequency of leukemia initiating cells was calculated using ELDA \cite{huELDAExtremeLimiting2009}.

\subsection{Human cell isolation from \glsentrylongpl{pdx}}

Cells from the injected femur, bone marrow, and spleen, were frozen viably after sacrifice.
Injected femur and bone marrow of mice engrafted with $> 10$ \% human cells were combined.
These cells were depleted of mouse cells using the Miltenyi Mouse Cell Depletion Kit (Miltenyi Biotec; samples with $> 20$ \% engraftment) or by cell sorting with human CD45 and human CD19 and/or CD34 cell surface antibodies to a purity of $> 90$ \%, as determined by post-processing flow cytometry.
Central nervous system cells from mice with $> 60$ \% engraftment were used directly for DNA isolation.
DNA was isolated using the QIAamp DNA Blood Mini or Micro Kit (Qiagen).

\subsection{Primary and \glsentrylong{pdx} sample sequencing}

\subsubsection{\Glsentrylong{rnaseq}}

\Gls{rnaseq} was performed as previously described \cite{dobsonRelapseFatedLatentDiagnosis2020}.
Briefly, amplified \gls{cdna} was sequenced as paired-end libraries on an Illumina HiSeq2000.
The libraries were sequenced as $2 \times 75$ bp for the adult and $2 \times 100$ bp for the pediatric samples.

\subsubsection{\Glsentrylong{mecapseq}}

\Gls{mecapseq} was performed using the SeqCapEpi CpGiant kit (Roche NimbleGen).
Briefly, the DNA library is prepared and bisulfite converted, amplified, and enriched using capture probes for targeted bisulfite-converted DNA fragments, then sequenced on a short-read sequencing machine.
More specifically, library preparation for \gls{mecapseq} was performed with the KAPA Library Preparation Kits, bisulfite conversion of genomic DNA was performed with the Zymo EZ DNA Methylation Lightning kit,
bisulfite-converted DNA libraries were amplified using the KAPA HiFi HotStart Uracil+ ReadyMix kit, and finally hybridized to probes from the SeqCap Epi Enrichment Kit.
Captured DNA fragments were sequenced on an Illumina HiSeq 2500 as $2 \times 125$ bp to a target depth of $70 \times 10^6$ read pairs per sample.

\subsubsection{\Glsentrylong{atacseq}}

Library preparation for \gls{atacseq} was performed with the Nextera DNA Sample Preparation Kit (FC-121-1030, Illumina), according to a previously reported protocol \cite{buenrostroTranspositionNativeChromatin2013}.
\Gls{atacseq} libraries were sequenced with an Illumina HiSeq 2500 sequencer to generate single-end 50 bp reads.

\subsection{Sequencing data analysis}

\subsubsection{Differential gene expression analysis}

The methods are described in \cite[REF][]{dobsonRelapseFatedLatentDiagnosis2020}.
Briefly, \gls{rnaseq} reads were aligned against the GRCh38 reference human genome with STAR (v2.5.2b) \cite{dobinSTARUltrafastUniversal2013} and annotated with the Ensembl reference (v90).
Default parameter were used with the following exceptions: chimeric segments were screened with a minimum size of 12 bp, junction overlap of 12 bp, and maximum segment reads gap of 3 bp; splice junction overlap of 10 bp; maximum gap between aligned mates of 100 000 bp; maximum aligned intron of 100 000; and alignSJstitchMismatchNmax of 5 1 5 5.
Transcript counts were obtained with HTSeq (v0.7.2) \cite{andersHTSeqPythonFramework2015}.
Data was library size normalized using the RLE method, followed by a variance stabilizing transformation using DESeq2 (v1.22.1) \cite{loveModeratedEstimationFold2014}.
Principal component analysis plots were generated on a per sample basis using the top 1 000 variable genes.
For downstream analysis, the mean expression of each sample clone condition was used.
For per-patient analyses, differentially expressed genes were identified between disease stage and clone status using DESeq2.
Genes with an \gls{fdr} $< 0.05$ and absolute $\log_2(\text{fold change}) > 1$ were considered significant.

\subsubsection{Identification of accessible chromatin peaks}

\Gls{atacseq} reads were aligned against the GRCh38 reference human genome with Bowtie2 (v2.0.5) \cite{langmeadFastGappedreadAlignment2012} with default parameters.
Accessible peaks were identified with MACS2 (v2.0.10) \cite{zhangModelbasedAnalysisChIPSeq2008} with the following command:

\begin{lstlisting}[basicstyle=\ttfamily]
  macs2 callpeak -f BED -g hs --keep-dup all -B --SPMR --nomodel --shift -75 --extsize 150 -p 0.01 --call-summits -n {sample_name} -t {input_bam}
\end{lstlisting}

A catalogue of peaks from all samples was collected with a custom R script.
\Gls{atacseq} signal was mapped from each sample to this catalogue using Bedtools \cite{quinlanBEDToolsSwissArmyTool2014} for downstream analysis.

\subsubsection{Bisulfite sequencing pre-processing}

Sequencing read qualities were assessed with FastQC (v0.11.8) \cite{simonandrewsFastQCQualityControl2010}.
Low quality bases were trimmed with Trim Galore! (v0.6.3) \cite{felixkruegerTrimGalore2012} with the following command:

\begin{lstlisting}[basicstyle=\ttfamily]
  trim_galore --gzip -q 30 --fastqc_args `-o TrimGalore' {sample_mate1} {sample_mate2}
\end{lstlisting}

Trimmed reads were aligned to the GRCh38 reference human genome with Bismark (v0.22.1) \cite{kruegerDNAMethylomeAnalysis2012} with default parameters.
Duplicates were removed from the resulting alignment file with the following command:

\begin{lstlisting}[basicstyle=\ttfamily]
  deduplicate_bismark -p --bam {input_bam}
\end{lstlisting}

The deduplicated BAM file was sorted by position with sambamba (v0.7.0) \cite{tarasovSambambaFastProcessing2015}.
$M$-biases were calculated with MethylDackel (v0.4.0) \cite{ryanMethylDackel2019}, and methylation $\beta$ values were extracted from the BAM files with the following command:

\begin{lstlisting}[basicstyle=\ttfamily]
  MethylDackel extract --mergeContext --OT 3,124,3,124 --OB 3,124,3,124 {ref_genome} {dedup_sorted_bam}
\end{lstlisting}

Both $M$ and $\beta$ values were for each \gls{cpg} were used in downstream analyses.

\subsubsection{Similarity network fusion}

Preprocessed data from each sample was collected with the following features: normalized gene expression abundance for all genes, chromatin accessibility signal within previously identified accessible peaks, and mean $\beta$ value for all \glspl{cpg} listed in the manifest for targeted bisulfite sequencing kit.
These features and sample labels were processed with the SNFtool R package \cite{wangSimilarityNetworkFusion2014} to perform the similarity network fusion analysis.
Graphs were constructed for all samples deriving from a single patient where each node is a sample and each edge is weighted according to the determined similarity between the samples.
Edges whose weights were below specific thresholds were removed from the graph.
The threshold weight for the fused graph was 0.05.
Similar graphs were constructed using the individual components for each sample (e.g. using just the similarity in \gls{rnaseq} data), and the component graphs were compared to the fused graph, to compare the importance of each feature.
Threshold weights for these individual graphs were determined to be $6 \times 10^{-5}$ for \gls{dname}, $4 \times 10^{-4}$ for gene expression, and $2 \times 10^{-4}$ for chromatin accessibility.

\subsubsection{\Glsentrylong{dmr} identification}

\Glspl{dmr} were identified using the dmrseq R package (v1.3.8) \cite{korthauerDetectionAccurateFalse2018} with an absolute filtering cutoff value of 0.05 and using the sequencing batch as an adjustment covariate.
Normal samples from all donors were compared pairwise based on their sorted cell type.
\Gls{ball} samples were compared by their designated disease stage (Dx, DRI, or Rel), and were compared both across all patients (e.g. all Dx samples against all Rel samples), or within a single patient (e.g. all Dx samples from Patient 1 against all Rel samples from Patient 1).
A multiple testing correction with the \gls{fdr} method was performed \cite{benjaminiControllingFalseDiscovery1995}.
Regions with an \gls{fdr} $< 0.1$ were determined to be significant.

\subsubsection{Gene ontology enrichment analysis}

Gene ontology enrichment analysis was performed using the PANTHER classification system (database version 2019-10-08) \cite{miLargescaleGeneFunction2013}.
Gene symbols for the genes whose promoter regions contained the recurrently hyper-methylated regions in all \gls{ball} patient samples were supplied, with the entire human genome as the background.
An over-representation Fisher test for biological processes was performed with an \gls{fdr} correction.
Biological processes at the top of the hierarchy with an \gls{fdr} $< 0.05$ were determined to be significant.
