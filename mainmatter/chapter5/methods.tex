\section{Methods}

\subsection{Patient selection and sample collection}

Cord blood pooling and isolation

B-ALL sample collection and cell sorting

\subsection{\glsentrylong{pdx} generation and \glsentrylong{lda}}

\subsection{Human cell isolation from \glsentrylongpl{pdx}}

\subsection{Primary and \glsentrylong{pdx} sample sequencing}

\subsubsection{\Glsentrylong{rnaseq}}

\subsubsection{\Glsentrylong{mecapseq}}

\subsubsection{\Glsentrylong{atacseq}}

\subsection{Sequencing data analysis}

\subsubsection{Differential gene expression analysis}

\subsubsection{Similarity network fusion}

Preprocessed data from each sample was collected with the following features: normalized gene expression abundance for all genes, chromatin accessibility signal within previously identified accessible peaks, and mean $\beta$ value for all \glspl{cpg} listed in the manifest for targeted bisulfite sequencing kit.
These features and sample labels were processed with the SNFtool R package \cite{wangSimilarityNetworkFusion2014} to perform the similarity network fusion analysis.
Graphs were constructed for all samples deriving from a single patient where each node is a sample and each edge is weighted according to the determined similarity between the samples.
Edges whose weights were below specific thresholds were removed from the graph.
The threshold weight for the fused graph was 0.05.
Similar graphs were constructed using the individual components for each sample (e.g. using just the similarity in \gls{rnaseq} data), and the component graphs were compared to the fused graph, to compare the importance of each feature.
Threshold weights for these individual graphs were determined to be $6 \times 10^{-5}$ for \gls{dname}, $4 \times 10^{-4}$ for gene expression, and $2 \times 10^{-4}$ for chromatin accessibility.

\subsubsection{\Glsentrylong{dmr} identification}

\Glspl{dmr} were identified using the dmrseq R package (v1.3.8) \cite{korthauerDetectionAccurateFalse2018} with an absolute filtering cutoff value of 0.05 and using the sequencing batch as an adjustment covariate.
Normal samples from all donors were compared pairwise based on their sorted cell type.
\Gls{ball} samples were compared by their designated disease stage (Dx, DRI, or Rel), and were compared both across all patients (e.g. all Dx samples against all Rel samples), or within a single patient (e.g. all Dx samples from Patient 1 against all Rel samples from Patient 1).
A multiple testing correction with the \gls{fdr} method was performed \cite{benjaminiControllingFalseDiscovery1995}.
Regions with an \gls{fdr} $< 0.1$ were determined to be significant.

\subsubsection{Gene ontology enrichment analysis}

Gene ontology enrichment analysis was performed using the PANTHER classification system (database version 2019-10-08) \cite{miLargescaleGeneFunction2013}.
Gene symbols for the genes whose promoter regions contained the recurrently hyper-methylated regions in all \gls{ball} patient samples were supplied, with the entire human genome as the background.
An over-representation Fisher test for biological processes was performed with an \gls{fdr} correction.
Biological processes at the top of the hierarchy with an \gls{fdr} $< 0.05$ were determined to be significant.
