\begin{abstract}

    Cancer results from aberrations at the molecular level that enable biological hallmarks.
    These aberrations can be found within the chromatin architecture of cancer cells that includes the genome, molecular modifications to the genome, and the three-dimensional organization of the chromatin fiber.
    The majority of genetic variants target non-coding regions of the genome and many genes affected by genetic and epigenetic variants have important roles in chromatin remodelling and maintenance.
    Thus, understanding the origins of cancer progression requires investigating the targets of these aberrations and how they impact the chromatin architecture.

    First, I investigated the impact of non-coding \glsentrylongpl{snv} that converge on \glsentrylongpl{cre} for the \emph{FOXA1} gene in primary prostate tumours.
    We found that deletion and repression of these \glsentrylongpl{cre} significantly decreases \emph{FOXA1} expression and \glsentrylong{pca} cell growth by altering the potential of \glsentrylongpl{tf} to bind at these loci.
    These results identify \glsentrylongpl{cre} that control \emph{FOXA1} expression in primary \glsentrylong{pca} as potential targets for therapeutic intervention.

    Secondly, I used chromatin conformation capture of 12 primary \glsentrylong{pca} tumours and 5 benign prostate tissues to characterize the three-dimensional genome organization.
    We found that large-scale organization, including \glsentrylongpl{tad} and compartments, is largely stable over oncogenesis but that small-scale focal chromatin interactions change between benign and tumour tissue.
    We also investigated the impact of \glsentrylongpl{sv} on chromatin organization and identify novel enhancer hijacking events.
    These results indicate that enhancer hijacking of \glsentrylong{pca} oncogenes may be a more common driver of disease than previously recognized.
    Then, I developed a statistical framework for differential gene expression analysis to address the impact of non-recurrent \glsentrylongpl{sv} in our primary prostate tumour cohort.
    This method improves on conventional gene expression fold change estimates in these unbalanced experimental designs.

    Finally, I investigated the genetic and epigenetic changes that underlie \glsentrylong{ball} relapse.
    I found recurrent loss of \glsentrylong{dname} in patient-matched relapse samples that indicate a more stem-like chromatin state.
    Together, my work investigates the relationship between multiple components of the chromatin architecture, and how aberrations to this architecture connects oncogenesis, disease progression, and relapse.

\end{abstract}
